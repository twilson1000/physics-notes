\documentclass[12pt, twoside]{article}
\usepackage{amsmath}

\begin{document}
	\section{Continuous transformation of O and X mode in cold plasma}
	Waves in cold plasmas with wave vector at an angle $\theta$ to the magnetic field are solutions to the equation
	\begin{equation}\label{wave_equation}
		\begin{pmatrix}
			S-N^2 \cos^2 \theta & -iD & N^2 \cos \theta \sin \theta \\
			iD & S-N^2 & 0 \\
			N^2 \cos \theta \sin \theta & 0 & P - N^2 \sin^2 \theta
		\end{pmatrix} \cdot
		\begin{pmatrix}
			E_x \\ E_y \\ E_z
		\end{pmatrix} = 0
	\end{equation}
	
	which uses two Stix parameters representing the normalised density and magnetic field
	\begin{equation}
		X_s = \left(\frac{\omega_{ps}}{\omega}\right)^2 \propto n_s \quad Y_s =  \frac{\omega_{cs}}{\omega} \propto B
	\end{equation}

	which are used to define 5 more parameters $P$ (parallel), $R$ (right), $L$ (left), $S$ (sum) and $D$ (difference)
	\begin{equation*}
		P = 1 - \sum_s X_s
	\end{equation*}
	\begin{equation*}
		R = 1 - \sum_s \frac{X_s}{1+Y_s}
	\end{equation*}
	\begin{equation*}
		L = 1 - \sum_s \frac{X_s}{1-Y_s}
	\end{equation*}
	\begin{equation*}
		S = 1 - \sum_s \frac{X_s}{1-Y_s^2} = \frac{R + L}{2}
	\end{equation*}
	\begin{equation*}
		D = \sum_s \frac{X_sY_s}{1-Y_s^2} = \frac{R - L}{2}
	\end{equation*}

	Taking the determinant of the matrix in \eqref{wave_equation} to vanish you get the Booker Quartic
	\begin{equation}\label{booker_quartic}
		AN^4 - BN^2 + C = 0
	\end{equation}
	with coefficients
	\begin{equation*}
		A = S \sin^2 \theta + P \cos^2 \theta
	\end{equation*}
	\begin{equation*}
		B = RL\sin^2 \theta + PS \left( 1+\cos^2 \theta \right)
	\end{equation*}
	\begin{equation*}
		C = PRL
	\end{equation*}
	
	Solving with the quadratic equation you can re-arrange the determinant $F$ to
	\begin{equation}
		F^2 = B^2 - 4AC = \left( RL - PS \right)^2 \sin^4 \theta + 4P^2 D^2 \cos^2 \theta
	\end{equation}

	giving a solution
	\begin{equation}\label{booker_quartic_solution}
		N^2 = \frac{B \pm F}{2A}
	\end{equation}

	There are a couple of different forms, if you divide \eqref{booker_quartic} by $\cos^2 \theta$ and re-arrange you get
	\begin{equation}\label{booker_quartic_with_tan}
		\tan^2 \theta = \frac{P \left(N^2 - R\right) \left(N^2 - L\right)}{\left(SN^2 - RL\right)\left(N^2 - P\right)}
	\end{equation}

	which is nice as it clearly shows the solutions in the cases of parallel and perpendicular propagation.
	
	If we only consider electron dynamics (high frequency) all the sums over species in the Stix parameters go away. In this case we drop the species subscript and just write $X$ and $Y$. Looking for solutions of \eqref{booker_quartic} of the form $N^2 = 1 - x$ and using the second form of the quadratic formula
	
	\begin{equation}
		x = \frac{2c}{-b \pm \sqrt{b^2 - 4ac}}
	\end{equation}

	after some tedious algebra expanding products of Stix parameters in terms of $X$ and $Y$ you get the Appleton-Hartree dispersion relation
	
	\begin{equation}\label{appleton_hartree}
		N^2 = 1 - \frac{2X\left(1-X\right))}{2\left(1-X\right)-Y^2\sin^2\theta \pm \sqrt{Y^4\sin^4 \theta + 4 \left(1-X\right)^2Y^2\cos^2\theta}}
	\end{equation}

	The Booker quartic \eqref{booker_quartic} and \eqref{booker_quartic_with_tan} are the same equation. The Appleton-Hartree dispersion is also the same equation but where we have restricted ourselves to high frequency.
	
	From these equations its clear there is a continuous transformation from $\theta=0$ to $\theta=\frac{\pi}{2}$ as the quadratic is a nice function. The coefficients of the Booker Quartic at $\theta=0$ are
	\begin{equation}
		A = P \quad B = 2PS \quad F = 2\left|PD\right|
	\end{equation}

	If we assume all terms are positive for now, from \eqref{booker_quartic_solution} we get (ignoring $P=0$ solution)
	\begin{equation}
		N^2 = \frac{2PS\pm2PD}{2P} = \begin{cases}
			S + D = R; \quad + \\ S - D = L; \quad -
		\end{cases}
	\end{equation}
	
	So we get the right handed or left handed circularly polarised wave for the positive and negative sign of the determinant respectively.
	
	The coefficients of the Booker Quartic at $\theta=\frac{\pi}{2}$ are
	\begin{equation}
		A = S \quad B = RL+PS \quad F = \left|RL-PS\right|
	\end{equation}
	
	Again assuming all terms are positive for now, from \eqref{booker_quartic_solution} we get
	\begin{equation}
		N^2 = \frac{RL+PS\pm \left(RL-PS\right)}{2S} = \begin{cases}
			\frac{RL}{S}; \quad + \\ P; \quad -
		\end{cases}
	\end{equation}

	Again we get the X mode or the O mode for the positive and negative sign of the determinant respectively. By fixing the sign we choose and continuously transforming from $\theta = 0$ to $\frac{\pi}{2}$ we get the right hand circularly polarised wave turns into the X mode and the left hand circularly polarised wave turns into the O mode.
	
	This mapping is not always true through. If we repeat this procedure but drop the assumption of positive definite we get
	\begin{equation}
		N^2 = \frac{2PS \pm 2\left|PD\right|}{2P} = \begin{cases}
			R; \quad P > 0, + \\ L; \quad P > 0, - \\
			L; \quad P < 0, + \\ R; \quad P < 0, -
		\end{cases} \quad \left(\theta=0\right)
	\end{equation}
	\begin{equation}
	N^2 = \frac{RL+PS \pm \left|RL-PS\right|}{2S} = \begin{cases}
		\frac{RL}{S}; \quad RL-PS > 0, + \\ P; \quad RL-PS > 0, - \\
		P; \quad RL-PS < 0, + \\ \frac{RL}{S}; \quad RL-PS < 0, -
	\end{cases} \quad \left(\theta=\frac{\pi}{2}\right)
	\end{equation}

	It is obvious $P = 1 - X$ changes sign at $X = 1$. $RL - PS$ is less obvious, if we expand it in the high frequency limit we get
	
	\begin{equation}
		RL - PS = \frac{-X Y^2 \left(1 + X\right)}{1 - Y^2}
	\end{equation}

	As $X > 0$ and $Y > 0$ the numerator is negative definite, meaning the only change in sign occurs at $Y = 1$. Therefore
	\begin{equation}
		\text{sign} \left(P\right) = \begin{cases}
			+1; \quad X < 1 \\ -1; \quad X > 1
		\end{cases}
	\end{equation}
	\begin{equation}
		\text{sign} \left(RL - PS\right) = \begin{cases}
			-1; \quad Y < 1 \\ +1; \quad Y > 1
		\end{cases}
	\end{equation}

	Combining this all together we get how the eigenmodes continuously transform in the four regions, arranging the table to match the CMA diagram
	\begin{center}
		\begin{tabular}{|c||c|c|}
			\hline $Y > 1$ & O$\rightarrow$L, X $\rightarrow$ R & X $\rightarrow$ L \\
			\hline $Y < 1$ & O$\rightarrow$R, X $\rightarrow$ L & X $\rightarrow$ R \\
			\hline \hline & $X < 1$ & $X > 1$ \\
			\hline
		\end{tabular}
	\end{center}

	Some of these transformations which are redundant have been omitted (O mode for $X > 1$).
	
	One notable implication of this is the possibility of converting from O to X mode at $X = 1$ and $\theta = 0$ as both waves continuously transform into a right hand circularly wave. The cold plasma approximation lacks the ability to describe mode conversion however so this is more of hand wave-y argument than anything mathematical.

\end{document}