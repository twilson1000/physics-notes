\section{Absorption of Electron Bernstein Waves}
\subsection{Doppler Shifted Resonance Condition}
Resonant absorption of electromagnetic waves in a magnetised plasma occurs when the oscillation of the wave electric field is synchronised with the gyration of the charged particles, in our case electrons. However, the finite value of the Larmor radius leads to absorption at higher harmonics. Bornatici has this to say on the matter \cite{bornatici1983}:

\textit{
	On purely classical grounds one would expect that here is a coupling to the electromagnetic field, i.e. emission and absorption, only at a wave frequency of $\omega=\omega_{ce}$. However, even if the electron dynamics is taken to be non-relativistic, there appear emission and absorption also around the harmonics $n\omega_{ce}$ because of the finiteness of the gyration (Larmor) radius of the electrons $\rho$ and of the speed of light, or, more precicely, of the quantity
}
	
\begin{equation}
k_{\perp}^2\rho^2=\left( \frac{v_{\perp}}{c} \frac{\omega}{\omega_{ce}} N \sin \theta \right)^2
\end{equation}

\textit{
	where $\theta$ is the angle between $\bm{k}$ and $\bm{B_0}$. In fact the finiteness of $k_{\perp}\rho$ has the consequence that the electromagnetic field generated by an electron gyrating with frequency $\omega_{ce}$, although periodic with a period $\frac{2\pi}{\omega_{ce}}$, is not simply harmonic, implying that higher Fourier components appear. Hence, this framework, although commonly termed `non-relativistic', actually contains a kind of relativistic effect already, due to the fact that Maxwell equations for the electromagnetic field are, indeed, relativistic equations.}
\textit{
	On the other hand, for emission and absorption of waves propagating close to perpendicularly to the magnetic field, it is just the relativistic electron dynamics that provides the dominant kinetic line-broadening mechanism, as for these directions the longitudinal Doppler effect tends to become small. Therefore, to correctly describe radiation at all angles a fully relativistic description is required to begin with, even when the electron velocity $v \ll c$.
}

Consider an electromagnetic wave propagating at frequency $\omega$ with wavevector $\bm{k}$. Taking the magnetic field along the $\bm{z}$ axis, the phase velocity of the wave parallel to the magnetic field $v_{\varphi,\parallel}$ is
\begin{equation}
	v_{\varphi,\parallel} = \frac{\omega}{k_{\parallel}}
\end{equation}
We can write the wave electric field along the magnetic field $\bm{E_{\parallel}}$, which is just a plane wave with wavevector $k_{\parallel}$ and frequency $\omega$
\begin{equation}
	\bm{E_{\parallel}} = E \sin \left( k_{\parallel}z - \omega t \right)
\end{equation}
In order to get resonance, the frequency of the electric field needs to match the period of the gyration. The components of the electron motion are
\begin{equation}
	x(t) = \rho \cos \left( -\omega_{ce} t \right)
\end{equation}
\begin{equation}
	y(t) = \rho \sin \left( -\omega_{ce} t \right)
\end{equation}
\begin{equation}
	z(t) = v_{\parallel} t
\end{equation}
where we include the minus sign to ensure $\omega_{ce} > 0$. The electric field in the frame of the particle is
\begin{equation}
	\bm{E_{\parallel}} = E \sin \left( k_{\parallel} v_{\parallel} t - \omega t \right) = E \sin \left( \left( k_{\parallel} v_{\parallel} - \omega \right) t \right)
\end{equation}
To be in resonance the frequency of the electric field must be a harmonic of the cyclotron frequency $-n \omega_{ce}$ where again we include the minus sign. We therefore get the resonance condition
\begin{equation}
	k_{\parallel} v_{\parallel} - \omega = -n \omega_{ce} \implies \omega = n \omega_{ce} + k_{\parallel} v_{\parallel}
\end{equation}

\subsection{Doppler Broadened Absorption Coefficient}
As particles have finite temperature they have a distribution of velocities. This means the Doppler shifted resonance condition can be satisfied by different parts of the distribution at different frequencies. A simple formula can be derived for this starting with the Doppler shifted resonance condition at the $n$-th harmonic, given wave perpendicular wavevector $k_{\parallel}$ and electron parallel velocity $v_{\parallel}$
\begin{equation}
	\omega = n \omega_{ce} + k_{\parallel} v_{\parallel}
\end{equation}
As we have finite temperature $v_{\parallel}$ is distributed according to a 1D Maxwell-Boltzmann distribution (Gaussian) centred on a velocity $v_0$ with width the electron thermal velocity $V_T$
\begin{equation}
	f \left(v_{\parallel} \right) = \left( \frac{1}{\pi v_T^2} \right)^{0.5} \exp \left( - \left( \frac{v_{\parallel} - v_0}{v_T} \right)^2 \right) \quad V_T = \sqrt{\frac{KT_e}{2m}}
\end{equation}

Now define the 'bulk' of the distribution as a velocity range where almost all of the particles are found. Standard results from Normal distributions tell us 99.7\% of particles are located within $3 \sigma = 3v_T$ of $v_0$. This distinction is somewhat arbitrary but it does successfully divide the distribution into a 'bulk' part containing almost all particles ($|v_{\parallel} - v_0| \le 3v_T$) and a 'tail' containing almost no particles ($|v_{\parallel} - v_0| > 3v_T$).

We can write the cold Doppler shifted resonant frequency $\omega_0 = n \omega_{ce} + k_{\parallel}v_0$. Assuming there is no absorption outside the bulk, we can write the minimum and maximum frequencies we expect absoption $\omega_1$ and $\omega_2$
\begin{equation}
	\omega_1=n \omega_{ce} + k_{\parallel} \left( v_0 - 3v_T\right) = \omega_0 - 3 k_{\parallel}v_T
\end{equation}
\begin{equation}
	\omega_2=n \omega_{ce} + k_{\parallel} \left( v_0 + 3v_T \right) = \omega_0 + 3 k_{\parallel}v_T
\end{equation}
This can be combined into a single equation for the minimum and maximum absorption frequency $\omega$ due to Doppler broadening relative to the cold Doppler shifted frequency $\omega_0$. Divide by $\omega$ and rearrange
\begin{equation}
	\omega = \omega_0 \pm 3 k_{\parallel}v_T \implies 1 = \frac{\omega_0}{\omega} \pm \frac{3 k_{\parallel}v_T}{\omega} = \frac{\omega_0}{\omega} \pm \frac{3 k_{\parallel}c}{\omega} \frac{v_T}{c} =  \frac{\omega_0}{\omega} \pm 3N_{\parallel}\beta
\end{equation}

where we introduce the normalised thermal velocity $\beta$. A useful approximation is $\beta \approx 9.953 \times 10^{-3} \sqrt{T_e\text{[eV]}}$. Cancel the factors of $2\pi$ to convert from angular frequency to frequency and rearrange. Take the inverse to get a formula for $f$. 
\begin{equation}\label{doppler_1}
	\frac{\omega_0}{\omega} = \frac{f_0}{f} = 1 \mp 3N_{\parallel}\beta \implies f = \frac{f_0}{1 \mp 3N_{\parallel}\beta}
\end{equation}

Here we stress the Doppler $\textbf{broadened}$ frequency band $f \in \left[ \frac{f_0}{1 - 3N_{\parallel}\beta}, \frac{f_0}{1 + 3N_{\parallel}\beta} \right]$ is relative to the Doppler $\textbf{shifted}$ frequency $f_0$ $\bf{not}$ the harmonic frequencies $nf_{ce}$, i.e. we have a shift on a shift. Calculating $f_0$ compared to $f_{ce}$ requires an estimate of the average electron parallel velocity $v_0$.

We can play a similar game using the cold Doppler shifted resonance condition, replacing $k_{\parallel}$ with $\frac{\omega_0 N_{\parallel}}{c}$
\begin{equation}\label{doppler_2}
	\omega_0 = n \omega_{ce} + \frac{\omega_0 N_{\parallel} v_0}{c} =  n \omega_{ce} + \omega_0 N_{\parallel} \beta_0 \quad \beta_0 := \frac{v_0}{c}
\end{equation}
We can once again cancel factors of $2\pi$ to convert to frequency. Rearranging we get an expression for $\omega_0$
\begin{equation}
	f_0 \left(1 - N_{\parallel} \beta_0 \right)=  n f_{ce} \implies f_0 = \frac{n f_{ce}}{1 - N_{\parallel} \beta_0}
\end{equation}

Substituting for $f_0$ in \eqref{doppler_1} using \eqref{doppler_2} we arrive the final expression for the Doppler shifted and Doppler broadened absorption frequencies limits around each cyclotron harmonic
\begin{equation}\label{doppler_final}
	f = \frac{nf_{ce}}{  \left( 1 - N_{\parallel} \beta_0 \right) \left( 1 \mp 3N_{\parallel}\beta \right) }
\end{equation}
The first bracket on the denominator is the Doppler shift term due to the electrons having relative velocity $v_0$ to the wave. If $v_0 = 0$ then $\beta_0 = 0$ we return to the cyclotron harmonics. The second bracket on the denominator is the Doppler broadening term due to finite temperature. Likewise if $T_e=0$ then $\beta = 0$ and we once again get the cyclotron harmonics.

\subsection{Absorption using a Complex Dispersion Relation}
Consider the characteristic decay of the intensity of an EBW $\bm{I}_{\text{EBW}}$. This has a characteristic decay rate per unit length along the ray trajectory $s$. We will call it $\alpha_{\omega}$
\begin{equation}
	I \sim I_0 \exp \left(-\alpha_{\omega} s \right)
\end{equation}
As $\bm{I} \sim \bm{E}^2$ we can write the decay rate of the electric field
\begin{equation}
	E \sim E_0 \exp \left( \frac{-\alpha_{\omega} \bm{s}}{2} \right)
\end{equation}

where $E_0$ is the electric field strength at the origin of the ray. We have already assumed the electric field has a plane wave form, so we can write
\begin{equation}
	E = E_0 \exp \left( i \bm{k} \cdot \bm{x} - \frac{\alpha_{\omega} s}{2} \right)
\end{equation}

This is equivalent to generalising the wavevector $\bm{k} \rightarrow \bm{k} + \frac{i \alpha_{\omega}}{2} \bm{s}$, where $\bm{s}$ is the unit vector pointing in the direction of the group velocity $\bm{v}_g$ i.e. $\bm{s} = \frac{\bm{v}_s}{|\bm{v}_s|}$ so the ray path length $s=\bm{s} \cdot \bm{x}$.
\begin{equation}
	i \bm{k} \cdot \bm{x} - \frac{\alpha_{\omega} s}{2} = i \bm{k} \cdot \bm{x} + i^2 \frac{\alpha_{\omega}}{2}\bm{s} \cdot \bm{x} = i \left(\bm{k} \cdot \bm{x} + \frac{i \alpha_{\omega}}{2} \bm{s}\right) \cdot \bm{x}
\end{equation}
To justify imaginary wavevectors we need to include the anti-Hermitian part of the dispersion tensor $\uubar{\Lambda} \left( \bm{k}, \omega \right)$. Our wave equation is
\begin{equation}
	\uubar{\Lambda}^h \left( \bm{k}, \omega \right) \cdot \bm{a} + i \uubar{\Lambda}^a \left( \bm{k}, \omega \right) \cdot \bm{a} = 0
\end{equation}

The real $\bm{k}$ solves the wave equation, i.e. $\uubar{\Lambda}^h \cdot \bm{a} = 0$. We can left multiply this expression by $\bm{a}$ to get a scalar
\begin{equation}\label{real_k_lambda_h}
	\bm{a}^{*} \cdot \uubar{\Lambda}^h \left( \bm{k}, \omega \right) \cdot \bm{a} := \Lambda_{aa}^h \left( \bm{k}, \omega \right) = 0
\end{equation}

Similarly we define $\Lambda_{aa}^h \left( \bm{k}, \omega \right) := \bm{a}^{*} \cdot \uubar{\Lambda}^a \left( \bm{k}, \omega \right) \cdot \bm{a}$. Our wave equation is therefore
\begin{equation} \label{absorb1}
	\Lambda_{aa}^h \left( \bm{k} + \frac{i \alpha_{\omega} \bm{s}}{2} \right) + i \Lambda_{aa}^a \left( \bm{k} + \frac{i \alpha_{\omega} \bm{s}}{2} \right) = 0
\end{equation}

If $\frac{\alpha_{\omega}}{2k} \ll 1$ we can Taylor expand to first order (making the evaulation at $\omega$ implicit)
\begin{equation}
	\Lambda_{aa}^h \left( \bm{k} + \frac{i \alpha_{\omega} \bm{s}}{2} \right) = \Lambda_{aa}^h \left( \bm{k} \right) + \frac{i \alpha_{\omega} \bm{s}}{2} \cdot \partial_{\bm{k}} \Lambda_{aa}^h \left( \bm{k} \right)
\end{equation}
\begin{equation}
	i\Lambda_{aa}^a \left( \bm{k} + \frac{i \alpha_{\omega} \bm{s}}{2} \right) = i\Lambda_{aa}^a \left( \bm{k} \right) - \frac{\alpha_{\omega} \bm{s}}{2} \cdot \partial_{\bm{k}} \Lambda_{aa}^a \left( \bm{k} \right)
\end{equation}

where $\partial_{\bm{k}} = \left( \frac{\partial}{\partial k_x}, \frac{\partial}{\partial k_y}, \frac{\partial}{\partial k_z} \right)$ is the partial derivatives over components of $\bm{k}$. Applying this to \eqref{absorb1}, understanding $\Lambda_{aa}^h$ and $\Lambda_{aa}^a$ are evaulated at $\left(\bm{k}, \omega \right)$

\begin{equation}
	\Lambda_{aa}^h + \frac{i \alpha_{\omega} \bm{s}}{2} \cdot \partial_{\bm{k}} \Lambda_{aa}^h + i \Lambda_{aa}^a = 0
\end{equation}

Immediately $\Lambda_{aa}^h = 0$ by definition from \eqref{real_k_lambda_h}. The first order Taylor expansion term of $\Lambda_{aa}^a$ has been omitted as it is real. If incorporated into $\alpha_{\omega}$ it would be an imaginary term, i.e. it's just a phase shift of $\bm{E}$. The second order Taylor expansion term would be real, however we are not expanding to that order (Note: This argument needs some clarification!). Rearranging we get an equation for $\alpha_{\omega}$
\begin{equation}\label{alpha_omega_1}
	\alpha_{\omega} = \frac{-2 \Lambda_{aa}^a}{\bm{s} \cdot \partial_{\bm{k}} \Lambda_{aa}^h}
\end{equation}

At high $N$ we are in the pure electrostatic regime. These modes are purely longitudinal, characterised by $\bm{a} \parallel \bm{k}$. This implies $\bm{a} = \alpha \bm{k}$ where $\alpha \in \Re$ is some scale factor. This also implies $\Lambda_{aa}^i=\alpha^2 \Lambda_{kk}^i$. Multiplying \eqref{alpha_omega_1} by $\alpha^2$ top and bottom allows us to rewrite as

\begin{equation}\label{alpha_omega_es_1}
	\alpha_{\omega} = \frac{-2 \Lambda_{kk}^a}{\bm{s} \cdot \partial_{\bm{k}} \Lambda_{kk}^h}
\end{equation}

